%! TEX program = lualatex

\documentclass[hidelinks,10pt]{article}

\usepackage[T1]{fontenc}
%\usepackage{luatextra}
\usepackage{fontspec}
\setmainfont{TeX Gyre Termes}
\usepackage{amsmath}
\usepackage{hyperref}
\usepackage{impnattypo}
\usepackage{mathtools}
\usepackage{microtype}

\usepackage[margin=1.25in]{geometry}

\newcommand{\HRule}{\rule{\linewidth}{0.5mm}}
\newcommand{\TODO}{\textbf{À décider}}

\pagenumbering{gobble}

\begin{document}


\begin{center}
    \large A language of geometry

    \normalsize [authors list]
\end{center}

\vspace{3\baselineskip}

\section{Introduction}

    Remark: This is an draft to be changed at will!

    In this document I will attempt to give the primitives of a minimalistic
    drawing program whose goal would be to provide relevant Minimal Description
    Length (MDL) in regard of what human may consider to be complex.

    I will thus introduce the required primitives, to move on to a formal
    description of a proto-language, syntax then semantics.

    Later on will be given examples in two categories: what we, as humans, had
    the language do, and what it can generate that surprised the author(s).

\section{The primitives}

    This part describes the primitives and why they were chosen/kept. Generally
    speaking, we want as few primitives as possible in order to have a "sane"
    language, but we also want to have something that we would imagine mimics
    human's behaviour.

    You have to imagine this as a reciepe of instruction given to a smart
    drawing system that has an internal state, the ability to integrate simple
    functions, and the ability to teleport back tro previous points in space
    \emph{and} time. Basically, a human hand and its owner's head.

    \subsection{\texttt{Save/Load}}

    These take a string and backup/restore the context/environement. This is
    usefull for branching scenario, in this language we never want "pen
    off"-like features as Logo had to be used as hacks to manually backtrack to
    a previous position.

    \subsection{\texttt{Turn}}

    This takes a float and rotates the direction in which the system
    is "looking". Note that it does not draw anything. Arguably this is a
    special case of a 1px movement of a high curvature circle at null speed and
    it may be removed later on.

    This is the only function that directly modifies the direction in which the
    system is looking, that I shall write later on $\theta$, and it does not
    set a specific value but rather updates it modulo $2\times\pi$. This is by
    design.

    \subsection{\texttt{Draw/Set}}

    Draw and Set are the two main function, they both change the internal
    values of the system but in quite different way:

    \begin{itemize}
     \item \texttt{Set} takes two floats, respectively $v$ and $\theta'$ and
         sets the internal values of this system to these values, where $v$
         represent the speed of drawing and $\theta'$ the general direction.
     \item \texttt{Draw} takes two floats, $v'$ and $\theta''$ and, then again,
         sets the internal value, where these two variables represent the
         derivative of the previously mentionned $v$ and $\theta'$
    \end{itemize}

    \texttt{Set} is intended to be used outside of any loop as it sets constant
    values, the other one is designed to be run within an integrate loop.

    \subsection{DiscreteRepeat}

    This takes a program and a number, and repeats the program the given amount
    of times. Repeat with variation, repeat with restart, and so on, are
    supposed to be handled through \texttt{Load/save}, thus the unique repeat
    function.

    \subsection{Integrate}

    This is the \emph{only} function actually putting something on the screen:
    in integrates the internall paramters across time and thus plot what needs
    to be plot. It takes a float, the length in an unspecified unit, and a
    program that will be repeated at each minimal instant.

    \subsection{\texttt{(;)}}

    This is required to ``concatenate'' programs.

    \subsection{Remark}

    I'm still not quite satisfied : Should integrate be the only drawing
    function? I think so. Should it take a programm as an argument and execute
    it? Not quite sure. This will most certainly change.

\section{Syntax}

\texttt{
    \begin{tabular}{rccl}
        Number$_{int}$ &$\Coloneqq$ &|&$0, 1, 2, ...$\\
        Number$_{float}$ &$\Coloneqq$ &|&$\pi, \sqrt{2}, 0.5, ...$\\
        Number &$\Coloneqq$ &|& Number$_{int}$ | Number$_{float}$ | Undefined
               &&&\\
        Program &$\Coloneqq$&|& Program ; Program \\
                &&|& Draw(Number$_{float}$, Number$_{float}$) \\
                &&|& Set(Number$_{float}$, Number$_{float}$) \\
                &&|& Save(string)\\
                &&|& Load(string)\\
                &&|& Turn(Number$_{float}$)\\
                &&|& DiscreteRepeat(Number$_{int}$) \{ Program \}\\
                &&|& Integrate(Number$_{float}$) \{ Program \}\\
                &&|& \{\}
    \end{tabular}
}

\section{Semantics}

    The formal semantic is not given yet although the "primitives" section
    should give hints as to what each function does. Note that you also have an
    operational semantics that is given by the actual code I'm also writing.

    Beware that for the time being the semantics of the number
    \texttt{Undefined} is not implemented yet.

\section{Bootstrap: what can we generate?}

\subsection{Tested}

\subsubsection{Square}

\begin{verbatim}
DiscreteRepeat(4) {
  Integrate(100.) { } ;
  Turn(pi/2)
}
\end{verbatim}

Repeat four time : draw a segment and turn.
The \texttt{Integrate} might be confusing but its good to know that at the time
being, the default values are $v=1, v'=0, \theta'=0,\theta''=0$ thus drawing a
straight line.

\subsubsection{Circle}
\begin{verbatim}
Set(1.,0.01) ;
Integrate(10000.) { }
\end{verbatim}

This sets a speed and a $\theta'$ and then draws it.

\subsubsection{Iso-Spiral}
\begin{verbatim}
Set(1.,0.1) ;
Integrate(10000.) {
    Draw(0.05,0.)
}
\end{verbatim}
\includegraphics[scale=0.2]{isospiral.png}

This sets an initial speed and an initial $\theta'$, then integrates while
accelerating thus drawing an isochronous spiral.

\subsubsection{Exponential Spiral}
\begin{verbatim}
Set(1.,0.) ;
Integrate(10000.) {
    Draw(0.,0.0001)
}
\end{verbatim}
\includegraphics[scale=0.2]{expspiral.png}

This is the other way around, the speed is kept constant but the rotation
changes. The values are arbitrary.

\subsubsection{Star --- Not Working, IMPORTANT}
\begin{verbatim}
Save("center") ;
DiscreteRepeat(10) {
    Integrate(100.) { } ;
    Turn(0.314) ;
    Load("center")
}
\end{verbatim}

This is not working for a good reason and I need to look deeper into it:
\texttt{Load} loads \emph{everything}, including the previous orientation, and
the loop gives us no counter to multiply/save a value, thus within the loop
there are no ways to do this currently.

\subsubsection{Person}
\begin{verbatim}
Save("Cou");
Set(2.,0.1);
Integrate(100.) {} ;
Load("Cou");
Turn(-1.57);
Integrate(25.) {};
Save("Bras");
Integrate(75.) {};
Save("Hanches");
Turn(-0.5);
Integrate(75.) {};
Load("Hanches");
Turn(0.5);
Integrate(75.) {};
Load("Bras");
Turn(1.57);
Integrate(50.) {};
Load("Bras");
Turn(-1.57);
Integrate(50.) {}
\end{verbatim}
\includegraphics[scale=0.5]{person.png}

Quite self explicit apart from the angles. Note that it is a bit lousy as my
value of $\frac{\pi}{2}$ is a bit rough.

\end{document}
