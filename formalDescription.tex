%! TEX program = lualatex

\documentclass[hidelinks,12pt]{article}                                         

\usepackage[T1]{fontenc}
%\usepackage{luatextra}
\usepackage{fontspec}
\setmainfont{TeX Gyre Termes}
\usepackage{amsmath}
\usepackage{hyperref}
\usepackage{impnattypo}
\usepackage{mathtools}
\usepackage{microtype}

\usepackage[margin=1.25in]{geometry}

\newcommand{\HRule}{\rule{\linewidth}{0.5mm}}
\newcommand{\TODO}{\textbf{À décider}}

\pagenumbering{gobble}

\begin{document}

\section{The primitives}

    The section is intended to be the list of the primitives and the
    reasoning behinf why we need them. A long list if you wish, with arguments
    for the existence of each element.

    \subsection{\texttt{Draw}}

    The \texttt{floats} are respectively a speed $s$, the variation of the
    radius of curvature $\theta$ --- i.e. how much the line is locally
    ``turning'' --- and the variation of this angle $\theta'$ --- its
    derivative.

    The point of this decomposition is to easily describe:
    \begin{enumerate}
        \item A segment or a line : $\theta = \theta' = 0$ and $v \neq 0$
            meaning that we are drawing straight.
        \item A circle, $v \neq 0$ \emph{and} $\theta \neq 0$, but $\theta' =
            0$, in this case we are moving and we have a fixed curvature, thus
            we are drawing a circular arc.
        \item A spiral with both $v$ and $\theta'$ set to non-zero values.
        \item \emph{Unclear whether this should be a feature or not} : $v =
            \theta' = 0$ to get angles?
    \end{enumerate}

    Draw alone will not do anything.

    \subsection{\texttt{Push/Pop}}

    Or any mechanism that could allow branching: continuations, mutable
    environment, etc.

    This would be required to have backtracking, e.g. in the diamond shape to
    avoid manually going back to the initial point.

    \subsection{\texttt{Face/Turn}}

    This is required in order to draw angles (or we could see this as a special
    case of \texttt{Draw} with $v = 0$ maybe?).

    Typical use-case: draw a triangle, you have to face a direction (absolute,
    less likely) or to turn (relative, easy) in order to draw the shape

    \subsection{\texttt{DiscreteRepeat}}

    Repeat a given set of instructions a given number of times.

    This is desireable for basic polygons, in particular highly regular one, to
    avoid instruction duplication.

    \subsection{\texttt{ContinuousRepeat} \emph{or} \texttt{integrate}}

    This is required in order to get \texttt{Draw} to actually do something.
    Draw alone will not do anything.

    \subsection{\texttt{(;)}}

    This is just required to ``concatenate'' programs.

\section{Syntax}

\texttt{
    \begin{tabular}{lll}
        Number $\Coloneqq$ &|&$0, 1, 2, ...$ \\
                           &|&$\pi, \sqrt{2}, 0.5, ...$ \\
                           &|&Undefined \\
                           &&\\
        Program $\Coloneqq$&|& Program ; Program \\
                           &|& Draw(Number, Number, Number) \\
                           &|& Save(string)\\
                           &|& Load(string)\\
                           &|& Face(Number)\\
                           &|& DiscreteRepeat(Number,Program)\\
                           &|& Integrate(Number,Program)\\
    \end{tabular}
}

\section{Semantics}

\section{Bootstrap: what can we generate?}

\end{document}
